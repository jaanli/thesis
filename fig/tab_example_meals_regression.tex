% !TEX root = ../main.tex
\newcommand\x{\scalebox{1.6}{\textbullet}}
\begin{table*}[t]
    \begin{center}
    \resizebox{0.9\textwidth}{!}{%
    \begin{tabular}{@{}lccccccccc|cccccc@{}}
      & \multicolumn{9}{c}{\bf Attributes} & \multicolumn{1}{c}{\bf                                                 User}\\
      \cmidrule(lr){2-10} \cmidrule(lr){11-15}
      {\bf Items} & Pizza & Eggs & Taco & Salad & Avocado & Chicken & Sardines & Beer & Coffee & 1  \\
      Morning Pizza & \x & \x & & & & & & & \x & \x \\
      Dinner Pizza & \x & & & & & & \x & \x & &   \\
      Small Salad & & & & \x & \x & & \x & & &  \\
      Big Salad & & \x & & \x & \x & \x & & & \x & \x  \\
      Taco & & & \x & & \x & \x & & & \x &  \\
      Fish Taco & & & \x & & & & \x & & &  \\
      \bottomrule
    \end{tabular}
    }
    \caption[Example binary classification data]{\label{tab:example-binary}Example binary classification data. A user consumes meals (rightmost column), and meals have attributes (table on the left. Meals are represented as datapoints $x_n$ with covariates being foods in the meals. The goal of a binary classifier trained on this data is to predict which meals a user will consume, or which datapoints $(x_n, y_n)$ have a label $y_n = 1$. An accurate classifier will information about which covariates are shared across positive or negative labeled datapoints; for example, this user consumes meals that include eggs and coffee.
    }
    \end{center}
\end{table*}